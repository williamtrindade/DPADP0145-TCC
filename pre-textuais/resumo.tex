% % % % %  RESUMO E PALAVRAS CHAVE DO RESUMO - OBRIGATORIO PARA MDT-UFSM
\resumo{
Escreva seu resumo aqui! Você pode digitá-lo diretamente neste arquivo ou usar o comando input. O resumo deve ter apenas uma página, desde o cabeçalho até as palavras chave. Caso seu resumo seja maior, use comandos para diminuir espaçamento e fonte (até um mínimo de 10pt) no texto.  Segundo a MDT, é preciso que os resumos tenham, no máximo, 250 palavras para trabalhos de conclusão de curso de graduação, pós-graduação e iniciação científica e até 500 palavras para dissertações e teses.
}
\palavrachave{Palavra Chave 1. Palavra 2. Palavra 3. (...)}
% "... deverão constar, no mínimo, três palavras-chave, iniciadas em
% letras maiúsculas, cada termo separado dos demais por ponto, e
% finalizadas também por ponto." MDT 2012

% % % % %  ABSTRACT E PALAVRAS CHAVE DO RESUMO - OBRIGATORIO PARA MDT-UFSM
\abstract{
Write your abstract here! As recomendações do resumo também se aplicam ao abstract. \lipsum[0-1]
}
\keywords{Keyword 1. Keyword 2. Keyword 3. (...)}