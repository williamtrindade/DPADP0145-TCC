\section{Arquitetura de software}
    
    \subsection{Arquitetura ou Design de software}
    
        \par Existe um mal entendimento a respeito dos termos arquitetura e design de software, apesar de terem diferenças, na prática não existe a necessidade de separa-los \cite{livro:martin:cleanarch}.
        
        \par O termo "arquitetura"\hspace{0.1cm} é usado no contexto de algo em alto nível e que não tem uma dependência dos detalhes de baixo nível. Todavia o termo "design"\hspace{0.1cm} parece muitas vezes sugerir as estruturas e decisões de baixo nível. Mas ao realizar uma análise nas arquiteturas de software na prática, perde-se o sentido de separá-los \cite{livro:martin:cleanarch}.
    
        \par Conforme descreve \citeonline[p. 10]{livro:martin:cleanarch} "Os detalhes de baixo nível e a estrutura de alto nível são partes do mesmo todo. Juntos, formam um tecido contínuo que define e molda o sistema".
        
        
    \subsection{Definição}
    
        \par Para uma melhor definição de arquitetura de software, segue conforme é definido por \citeonline[p. 10]{livro:martin:cleanarch} "A arquitetura de um sistema de software é a forma dada a esse sistema pelos seus criadores". A forma descrita pelo autor, está relacionada com a organização dos componentes ao qual o software é dividido e na maneira que esses componentes comunicam-se entre si \cite{livro:martin:cleanarch}.
        
        \par A arquitetura de software está altamente conectada com a manutenibilidade do software \cite{artigo:dantas:2021}. Além disto, a arquitetura de software seja ela qual for, exerce grande influência sobre os requisitos não-funcionais de um software \cite{artigo:lopes:2021}.
    
    \subsection{Propósito da arquitetura de software}
        \par O propósito primário da arquitetura é suportar o ciclo de vida do sistema. Uma boa arquitetura torna o sistema fácil de entender, desenvolver, manter e implantar. O objetivo final é minimizar o custo da vida útil do sistema e não maximizar a produtividade do programador \cite{livro:martin:cleanarch}.


    \subsection{Tipos de arquiteturas de software}
        \subsubsection{Arquitetura em camadas}
            \par Uma arquitetura em camadas é um modelo de organização de componentes de software onde o projeto é separado em camadas horizontais, cada camada tem seu papel específico na aplicação. Não é especificado um número especifico de camadas nessa arquitetura. Esse tipo de arquitetura beneficia todo o ciclo de vida do software \cite{artigo:bueno:2021}.

\section{Paradigmas de programação}
    \subsection{Programação estruturada}
    \subsection{Programação funcional}
    \subsection{Programação orientada a objetos}

\section{Principios de design de software}
    \subsection{SRP: Princípio da Responsabilidade Única (Single Responsibility Principle)}
    \subsection{OCP: Princípio do Aberto/Fechado (Open-Closed Principle)}
    \subsection{LSP: Princípio de Substituição de Liskov (Liskov Substitution Principle)}
    \subsection{ISP: Princípio da Segregação de Interface (Interface Segregation Principle)}
    \subsection{DIP: Princípio da Inversão de Dependência (Dependency Inversion Principle)}

\section{Arquitetura limpa}