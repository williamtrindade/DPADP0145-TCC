O livro Arquitetura Limpa de Robert C. Martin (2020) oferece insights complementares que reforçam a relevância dos conceitos apresentados. Martin enfatiza que "o objetivo principal de uma boa arquitetura é suportar o ciclo de vida do sistema" (Martin, 2020, p. 197), destacando que a Arquitetura Limpa deve ser centrada em casos de uso para refletir o propósito do sistema, e não em frameworks ou ferramentas específicas. Ele argumenta que "as boas arquiteturas são aquelas que permitem aos arquitetos descrever com segurança as estruturas que suportam os casos de uso, sem se comprometer com frameworks, ferramentas e ambientes" (Martin, 2020, p. 197). Essa visão é particularmente relevante para o projeto SmartEye, onde a separação de camadas permite a integração de microsserviços sem dependência de tecnologias específicas (Dantas et al., 2021, p. 6).

Outro ponto destacado por Martin é a importância de limites arquiteturais claros, que "separam os componentes do sistema de forma que as mudanças em um componente não exijam mudanças em outros" (Martin, 2020, p. 229). Ele sugere avaliar os custos de implementar esses limites, decidindo quais devem ser totalmente implementados, parcialmente implementados ou ignorados, com base no contexto do projeto (Martin, 2020, p. 229). Essa abordagem pragmática pode ser aplicada ao projeto SmartEye, onde a escolha de microsserviços com limites bem definidos reduz o acoplamento, mas exige planejamento para evitar sobrecarga (Dantas et al., 2021, p. 6). Esses conceitos reforçam a necessidade de equilíbrio entre complexidade e funcionalidade na aplicação da Arquitetura Limpa.