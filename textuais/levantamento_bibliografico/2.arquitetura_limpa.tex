\section{Princípios de Arquitetura Limpa}
    \subsection{Contexto histórico}
    
    \subsection{Definição e objetivos}
        \par A arquitetura limpa, é um conjunto de princípios e práticas de design de software que visa construir sistemas estruturados, flexíveis e de fácil manutenção, reduzindo os recursos humanos necessários para seu desenvolvimento e evolução ao longo do tempo \cite{livro:martin:cleanarch}. Esse conceito prioriza a criação de software "suave" (soft), projetado para facilitar adaptações às mudanças nos requisitos dos stakeholders, de modo que o esforço para implementar alterações seja proporcional apenas ao escopo da mudança, e não à sua forma \cite{livro:martin:cleanarch}.
  
        \par A arquitetura limpa mitiga a complexidade acidental, promovendo sistemas que permanecem funcionais e econômicos, evitando o aumento desproporcional de custos observado em projetos mal estruturados \cite{livro:martin:cleanarch}.

    
        \par Um dos alicerces da arquitetura limpa é a Regra da Dependência, que estabelece que as dependências entre componentes de software devem sempre apontar para as políticas de alto nível, ou seja, as regras de negócio centrais do sistema (p. 22). Essa regra garante a separação entre a lógica principal e os detalhes de implementação, como frameworks, bancos de dados ou interfaces de usuário, que são tratados como elementos secundários e substituíveis (p. 26). Essa abordagem permite que a arquitetura limpa crie sistemas modulares e testáveis, capazes de se adaptar a mudanças tecnológicas sem comprometer a funcionalidade principal, como ilustrado nos exemplos de camadas e limites descritos por Martin (p. 25-26).

        \par A arquitetura limpa também é concebida como um processo contínuo de investigação, descrito como uma "hipótese que precisa ser comprovada por implementação e medição" (p. 33). Martin enfatiza que os desenvolvedores trabalham com conhecimento incompleto, e a arquitetura limpa deve ser flexível para incorporar novas descobertas e requisitos (p. 33). Esse princípio é crucial para evitar sistemas rígidos ou excessivamente complexos, como no estudo de caso apresentado, onde a ausência de uma estrutura limpa levou a uma queda drástica na produtividade e a custos insustentáveis devido à acumulação de código desorganizado (p. 50-52).

        \par Por fim, a arquitetura limpa incorpora princípios de design, como o Princípio da Responsabilidade Única (SRP), o Princípio Aberto/Fechado (OCP) e o Princípio da Inversão de Dependência (DIP), que asseguram a criação de componentes coesos e desacoplados (p. 24). Esses princípios, alinhados a paradigmas como a programação orientada a objetos e funcional, organizam o software de forma intuitiva, facilitando sua manutenção e evolução (p. 23-24). Segundo Martin, a aplicação rigorosa da arquitetura limpa não apenas eleva a qualidade técnica do sistema, mas também reduz a frustração da equipe de desenvolvimento e aumenta a confiança dos stakeholders, contribuindo para a sustentabilidade e o sucesso do projeto (p. 44).