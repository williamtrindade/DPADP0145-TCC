\chapter{Introdução}
    \section{Contextualização}
      \par A Arquitetura Limpa, proposta por Robert C. Martin, surge como uma resposta à complexidade crescente no desenvolvimento de software. Em um cenário onde a tecnologia evolui rapidamente, a necessidade de criar sistemas que sejam robustos, testáveis e de fácil manutenção se torna essencial. A Arquitetura Limpa oferece uma abordagem que visa desacoplar componentes do sistema, permitindo que mudanças em uma parte do código não impactem outras, facilitando assim a evolução e a adaptação do software às novas demandas.

    \section{Problema}
    
     \par Apesar das vantagens potencialmente significativas da Arquitetura Limpa, muitos desenvolvedores e equipes de software enfrentam desafios ao adotá-la em suas práticas de desenvolvimento. Dificuldades como a resistência à mudança, a curva de aprendizado necessária para compreender seus princípios e a aplicação prática em projetos existentes são comuns. Esse cenário levanta questões importantes sobre a real eficácia da Arquitetura Limpa em diferentes contextos de desenvolvimento.

    \par Diante disso, este trabalho busca investigar: Quais são os fatores que influenciam a adoção e a implementação bem-sucedida da Arquitetura Limpa em projetos de software? Além disso, até que ponto a Arquitetura Limpa se mostra vantajosa em comparação com outras abordagens arquiteturais, considerando aspectos como manutenibilidade, escalabilidade e eficiência no desenvolvimento?

    \section{Objetivos}
        \subsection{Objetivo Geral}
    
            \par Este trabalho tem como objetivo realizar um estudo sobre a Arquitetura Limpa (Clean Architecture) com base na literatura existente. Busca-se explorar os princípios e fundamentos que a sustentam, assim como os padrões de design relevantes para sua implementação. Além disso, serão discutidos os principais fatores que motivam a sua adoção em projetos de software, destacando suas vantagens e desafios.

        \subsection{Objetivos Específicos}
            \par Para a realização deste trabalho de conclusão de curso serão buscados os seguintes objetivos específicos:

            \begin{itemize}

                \item Realizar uma revisão da literatura sobre os princípios fundamentais da Arquitetura Limpa.

                \item Analisar padrões de design que suportam a implementação da Arquitetura Limpa em sistemas de software.

                \item Investigar as vantagens e desafios da aplicação da Arquitetura Limpa em projetos reais.

                \item Identificar os fatores que motivam ou não a adoção da Arquitetura Limpa em cenários de desenvolvimento de software.

            \end{itemize}
            
    \section{Metodologia}
        \par A metodologia utilizada neste trabalho de conclusão de curso tem natureza básica pura, com o objetivo de realizar um estudo descritivo, utilizando de uma abordagem qualitativa, para a obtenção dos dados do presente trabalho foi realizada uma pesquisa bibliográfica como procedimento técnico.
        